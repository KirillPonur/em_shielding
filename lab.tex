\input{text/preamble}
\newcommand{\vH}{\textbf{H}}
\newcommand{\vE}{\textbf{E}}
\newcommand{\vB}{\textbf{B}}
\newcommand{\vD}{\textbf{D}}
\newcommand{\vr}{\textbf{r}}
\newcommand{\vj}{\textbf{j}}
\newcommand{\vk}{\textbf{k}}
\newcommand{\vx}{\textbf{x}}
\newcommand{\vy}{\textbf{y}}
\newcommand{\vz}{\textbf{z}}
\usepackage{mathtools}
\mathtoolsset{showonlyrefs=true}
\begin{document}

\def\labauthors{Войтович Д.А., Понур К.А.}
\def\labgroup{440}
\def\labnumber{1}
\def\labtheme{Электромагнитное экранирование}
\def\department{Кафедра электродинамики}
\input{text/titlepage}


\newpage

{\bfseries Цель работы:} 
Экспериментальное наблюдение явления экранирования переменного магнитного поля металлическими оболочками и выяснение роли основных физических факторов (свойств материала экрана, а именно - проводимости и магнитной проницаемости; толщины его стенок; частоты поля), определяющих степень проникновения поля через экран, а также теоретический расчет экранирующих свойств металлических оболочек на простой модели и сопоставление экспериментальных и теоретических данных.

\section{Теоретическая часть}
\subsection{Введение}

Под электромагнитным экранированием понимается изоляция некоторой области пространства от проникновения электромагнитных полей, существующих в соседних областях. В статических или переменных квазистационарных полях (которым соответствуют длины волн, много большие характерных размеров используемых приборов и устройств) такая изоляция осуществляется обычно с помощью замкнутых металлических оболочек - экранов. 

Общей физической причиной ослабления поля внутри экрана является то обстоятельство, что наведенные в нем внешнем полем токи (или заряды) создают во внутренней области поле, противоположное внешнему. В результате суммарное поле в этой области, складывающееся из полей внешних и наведенных источников, уменьшается. 
% Если $\lambda_0\gg l$, где $\lambda_0$ -- длина волны экранируемого поля, $l$ -- характерный размер экранируемой области, то такая изоляция осуществляется обычно с помощью замкнутых металлических оболочек -- экранов.} переменного магнитного поля стальными ($\sigma \simeq 0.7\cdot 10^{17} \,c^{-1}, \mu \sim 10^2 \divisionsymbol 10^3 $ при $H \sim 10$ эрстед) и латунными ($\sigma \simeq 1.5\cdot 10^{17}\, c^{-1}, \mu \cong 1$) цилиндрическими экранами на частотах экранируемого поля $20\divisionsymbol10^4$ Гц.  

% Внутренние размеры всех экранов одинаковы (высота и радиус основания $h=R$ = 5 см), а толщина стенок различна ($d=$ 0.2 см, 0.5 см, 1 см). 

% Строгий аналитический расчет экранирования цилиндрическими замкнутыми экранами невозможен. Простыми моделями, допускающими точное решение задачи в аналитических функциях являются, например, модель плоского, цилиндрического и сферического слоя.

% Среди этих моделей замкнутый экран можно описать только сферическим слоем. Кроме того, характерные размеры используемого экрана малы по сравнению с длиной волны экранируемого поля ($\lambda_0\sim10$ км $\Rightarrow h,D\ll\lambda_0$). Поэтому адекватным выбором для качественных оценок является именно сферическая модель. Для того, чтобы получить близкие и количественные результаты, логично взять для сферической модели же объем внутренней полости, что и у используемого экрана, и ту же толщину стенки.


% Если замкнутая однородная сферическая оболочка помещена в квазистатическое внешнее поле с комплексным вектором напряженности $\vec{H}_{0} e^{i \omega t}$, которое в ее отсутствие является однородным, то поле в ограничиваемой ею области $\vec{H}_{1} e^{i \omega t}$ также однородно. Эффективность экранирования удобно характеризовать величиной отношения комплексных амплитуд этих полей:
% \begin{equation} 
% 	\eta_{m}=\frac{H_0}{H_1}
% 	\label{eq:1}
% \end{equation}
% Безразмерная величина $|\eta_{m}|$ показывает, в какое число раз ослабляется поле в экранированной области, и может быть названа \textbf{коэффициентом ослабления}. 

% По результатам эксперимента вычисляется коэффициент ослабления для всех экранов на всех частотах указанного диапазона, и сравнивается с теоретическим результатом для модели сферического слоя.

\subsection{Расчет экранирующего действия металлических оболочек}

В работе используются оболочки цилиндрической формы. Для получения качественных оценок ослабления поля в экранированной области и установления характера его зависимости от параметров можно ограничиться изучением более простых моделей, допускающих точное решение задачи в известных аналитических функциях. Поскольку высота и диаметр внутренней полости используемых в работе экранирующих цилиндров одинаковы и весьма малы по сравнению с длиной волны в свободном пространстве $lambda_0$, наиболее подходящей моделью следует считать сферический слой, который имеет тот же объем внутренней полости и внешний радиус $a \ll \lambda_0$. Последнее условие означает, что вне металла (как во внешней, так и в экранируемой областях) поле можно рассматривать как квазистатическое. Приведем основные результаты решения задачи об экранирующих свойствах сферического слоя по отношению к переменному магнитному полю. 

Если замкнутая однородная сферическая оболочка помещена в квазистатическое внешнее поле с комплексным вектором напряженности $\vec{H}_{0} e^{i \omega t}$, которое в ее отсутствие является однородным, то поле в ограничиваемой ею области $\vec{H}_{1} e^{i \omega t}$ также однородно. Эффективность экранирования удобно характеризовать величиной отношения комплексных амплитуд этих полей:
\begin{equation} 
	\eta_{m}=\frac{H_0}{H_1}
	\label{eq:1}
\end{equation}
Безразмерная величина $|\eta_{m}|$ показывает, в какое число раз ослабляется поле в экранированной области, и может быть названа \textbf{коэффициентом ослабления}. Она сильно зависит от соотношения между толщиной экрана $d$ и толщиной скин-слоя $\delta=\frac{c}{\sqrt{2\pi\sigma\mu\omega}}$ ($c$ - скорость света в вакууме, $\sigma$ - проводимость, $\mu$ - магнитная проницаемость экрана). Рассмотрим два предельных случая:

% \paragraph{Постановка задачи.} Рассмотрим однородный сферический слой внешним радиусом $a$, толщиной $d$. Считаем, что $a\ll\lambda_0$. Выкладки производятся в сферической системе координат $(r,\theta,\phi)$, где полярная ось выбрана параллельно внешнему полю $\vec{H}_0$.

% Задача разбивается на три области:
% \begin{equation}
% 	\left\{\begin{aligned}
% 		\epsilon=\mu=1, k=k_0 &\quad\text{при}\quad r<a-d\\
% 		k=k_0\sqrt{\epsilon\mu} &\quad\text{при}\quad  a-d \geq r\leq a\\
% 		\epsilon=\mu=1, k=k_0 &\quad\text{при}\quad r>a\\
% 	\end{aligned}\right.
% \end{equation}

% \paragraph{Определение вида $\vec{A}(\vec{H}_0)$.} Значения полей $\vec{A},\vec{B},\vec{H}$ ( где $\vec{B}=\mu\vec{H}=\mathrm{rot}\,\vec{A}$) всюду должны полностью определяться вектором $\vec{H}_0$. Так как поля $\vec{B},\vec{H},\vec{H_0}$ - псевдовектора, а вектор $\vec{A}$ -- истинный вектор, то зависимость $\vec{A}(\vec{H}_0)$ можно получить только векторным произведением (векторное произведение псевдовектора на истинный дает истинный вектор):
% \begin{equation}
% 	\vec{A}=\vec{F}\times\vec{H}_0
% \end{equation}
% В силу отсутствия выделенных направлений, кроме $\vec{H}_0$, можно предположить радиальность $\vec{F}$, т.е. 
% \begin{equation}
% 	\vec{F}=\vec{r}_0 F(r)=\nabla f(r)
% \end{equation}
% Подставляя вектор $\vec{A}$, выраженный через $f(r)$ в уравнение Гельмгольца $\Delta\vec{A}+k^2\vec{A}=0$, получим
% \begin{equation}
% 	\Delta f +k^2f=0
% \end{equation}
% Общее решение такого уравнения известно:
% \begin{equation}
% 	f(r)=C_1\frac{e^{ikr}}{r}+C_2\frac{e^{-ikr}}{r}
% \end{equation}
% Здесь $\pm$ в экспоненте определяет расходящуюся и сходящуюся сферические волны.

% \paragraph{Выражение $A_\phi,B_r,B_\phi$ из $\vec{A}(\vec{H}_0)$.} Подставляя найденную в предыдущем пункте зависимость $\vec{A}(\vec{H}_0)$ в выражения для полей и проецируя на оси, получим выражения для $A_\phi,B_r,B_\phi$ (через константы $C_1$ и $C_2$). Эти формулы получены из общих соображений и должны давать значение поля в сферическом слое.

% \paragraph{Решение уравнения Гельмгольца в зонах квазистатики.} Рассмотрим решение задачи в первой (во внутренности сферического слоя) и третей (вне слоя) областях. В силу наложенного условия $a\ll\lambda_0\Rightarrow k_0a\ll 1$, в этих областях поле имеет квазистатический характер, и уравнение Гельмгольца упрощается до $\Delta\vec{A}=0$, откуда следует $\Delta f =\mathrm{const}$. 

% Общее решение такого уравнения $f(r)=A_1r^2+A_2r^{-1}$. Подставив его в выведенные ранее формулы для $\vec{A}(f)$ и вычисляя ротор $\vec{A}$, получаются выражения в квазистатическом пределе для $A_\phi,B_r,B_\phi$ через константы $A_1, A_2$. Эти формулы верны только в первой и третей областях, причем лишь не очень далеко от слоя (выполнение условия квазистатики $k_0r\ll 1$)


% \paragraph{Предельные и граничные условия.} Нужно наложить условия конечности поля в точке $r=0$, а также стремление к внешнему полю $\vec{H}_0$ при удалении от экрана, на формулы в зонах квазистатики. Эти два предельных условия позволяют избавиться от двух констант в этих зонах. Важным результатом будет также то, что константа $A_1$ определяет отношение амплитуд полей ($\eta_m=H_0/H_1=-\frac14A_1$)

% Далее, необходимо соблюсти граничные условия из условий непрерывности нормальной компоненты $B_r$ на границах областей $r=a$ и $r=a-d$, откуда получается система четырех линейных алгебраических уравнений относительно $A_1,A_2,C_1,C_2$. Из этой системы можно выразить $A_1$, а значит, и $\eta_m$:
% \begin{equation}
% 	\eta _ { m } = \left( 6 i \mu k ^ { 3 } a ^ { 3 } \right) ^ { - 1 } \left( F _ { + } e ^ { i k d } - F _ { - } e ^ { - i k d } \right)
% 	\label{eq:etaF}
% \end{equation}
% где
% \begin{gather}
% 	 F _ { \pm } = 2 \mu ^ { 2 } ( 1 \mp i k a ) ( 1 \pm i k b ) +   \mu \left[ ( 1 \pm i k a ) \left( 1 \pm i k b - k ^ { 2 } b ^ { 2 } \right) - 2 ( 1 \pm i k b ) \left( 1 \mp i k a - k ^ { 2 } a ^ { 2 } \right) \right] - \\ - \left( 1 \mp i k a - k ^ { 2 } a ^ { 2 } \right) \left( 1 \pm i k b - k ^ { 2 } b ^ { 2 } \right) 
% \end{gather}
% По определению, \textbf{толщина скин-слоя} $\delta$
% \begin{equation}
% 	\delta=\frac{c}{\sqrt{2\pi\sigma\mu\omega}}
% \end{equation}
% Для металлов вплоть до частот оптического диапазона
% \begin{equation}
% 	k=\frac{1-i}{\delta}
% \end{equation}
% В двух предельных случаях ($\delta \ll d$  и $\delta \gg d$) выражение для $\eta_{m}$(в общем случае довольно громоздкое) существенно упрощается, принимая также во внимание дополнительное условие $d \ll a$.

В пределе $\delta \ll d$ (сильный скин-эффект)
	\begin{equation} 
		\eta_{m}=\frac{1}{6}\left[(1-i) \frac{\mu \delta}{a}+3+(1+i) \frac{a}{\mu \delta}\right] \exp \left[(1+i) \frac{d}{\delta}\right]
	\label{eq:2}
	\end{equation}

При $\mu=1$
\begin{equation} 
	\eta_{m}=\frac{1}{6}(1+i) \frac{a}{\delta} \exp \left[(1+i) \frac{d}{\delta}\right]
	\label{eq:3}
\end{equation}

Область отсутствия скин-эффекта (в пределе $\delta \gg d$):
\begin{equation}
	\eta_{m}=1+\frac{2}{3}\frac{d}{a}\frac{(\mu-1)^2}{\mu}+i\frac{2}{3}\frac{ad}{\mu \delta^2}
	\label{eq:4}
\end{equation}
При $\mu=1$
\begin{equation} 
	\eta_{m}=1+i \frac{2 a d}{3 \delta^{2}}
	\label{eq:5}
\end{equation}

Для приближенных оценок величины $\eta_{m}$ (с точностью $\sim10\%$) выражения \eqref{eq:2}—\eqref{eq:5} можно использовать и в промежуточном случае ($\delta \simeq d$), разграничивая области применимости формул \eqref{eq:2}, \eqref{eq:3}, с одной стороны, и \eqref{eq:4}, \eqref{eq:5}, с другой стороны, точкой $\delta = d$.

\newpage
\section{Экспериментальная часть}
Лабораторная установка предусматривает проведение измерений коэффициентов ослабления для трех латунных и трех стальных экранов цилиндрической формы. 

Схема измерения $|\eta_m|$ заключалась в следующем: переменное магнитное поле создается внутри соленоида, подключенного к выходу генератора. Внутренние размеры всех экранов одинаковы (высота и радиус основания $h=R$ = 5 см), а толщина стенок различна ($d=$ 0.2 см, 0.5 см, 1 см). 

Сталь: $\sigma \simeq 0.7\cdot 10^{17} \,c^{-1}, \mu \sim 10^2 \divisionsymbol 10^3 $ при $H \sim 10$ эрстед.

Латунь: $\sigma \simeq 1.5\cdot 10^{17}\, c^{-1}, \mu \cong 1$ при $H \sim 10$ эрстед.

Схема установки:
\begin{figure}[H]
    \centering
    \includegraphics[width = 0.9\linewidth]{imgs/graphs/img744.jpg}
    \caption{Схема установки}
    \label{fig:1}
\end{figure}

Переменное магнитное поле создается внутри соленоида, подключенного к выходу звукового генератора. В качестве индикатора
поля используется второй соленоид, с выхода которого переменное напряжение может подаваться на усилитель вольтметра.
Надевая больший (генераторный) соленоид сначала на открытый (неэкранированный) индикатор, а затем на индикатор,
закрываемый экраном, и измеряя, как изменяются при этом показания вольтметра, можно (при неизменности амплитуды тока в
цепи внешнего соленоида) определить коэффициент ослабления. Поскольку внесение металлического экрана внутрь внешнего
соленоида изменяет его коэффициент самоиндукции, а следовательно, и его импеданс, сила тока в цепи внешнего соленоида и
создаваемое этим током магнитное поле $\vH_0$ при наличии экрана и в его отсутствие могут быть различными. Это нужно
учитывать. В используемой схеме предусмотрено измерение относительных изменений токов как во внутреннем, так и во
внешнем соленоидах. С этой целью в цепь внешнего соленоида введено сопротивление $R$, напряжение с которого подается на
вертикальный усилитель осциллографа. Тогда:
\begin{equation}
 	|\eta_m|=\frac{V_0U_e}{V_eU_0},
	\label{eq:7}
\end{equation}
где V и U - соответственно показания вольтметра и осциллографа, индексы o и e относятся соответственно к величинам, измеренным без экрана и с экраном. 

% В качестве индикатора поля используется второй соленоид меньших размеров (индикаторный), с выхода которого переменное напряжение подается на усилитель вольтметра. 

% Надевая генераторный соленоид сначала на неэкранированный индикаторный, а затем на индикаторный соленоид, закрываемый экраном, и измеряя, как изменяются при этом показания вольтметра, можно по ним вычислить коэффициент ослабления $|\eta_m|$. Кроме этого, необходимо также измерять амплитуду напряжения на осциллографе, подключенного к сопротивлению, которое стоит последовательно генераторному соленоиду.

% \subsection{Учет в $|\eta_m|$ искажения $L_{gen}$ экраном}

% Поле в соленоиде пропорционально току, который в нем течет: $H\sim I$ (это нетрудно вывести на примере бесконечного соленоида с непрерывной обмоткой). Прикладывая к генераторному соленоиду напряжение постоянной амплитуды $u_0$, мы получаем
% \begin{equation}
% 	u_0=Z\cdot I=i\omega L\cdot I \quad\Rightarrow\quad H_{ext}\sim \frac{u_0}{\omega L}
% \end{equation}

% Когда мы измеряем напряжение на индикаторном соленоиде, находящемся в поле $H_{in}$, оно равно 
% \begin{equation}
% 	V=i\omega L_{ind}\cdot I_{ind} \sim \omega  L_{ind} H_{in}
% \end{equation}

% Индуктивность индикаторного соленоида не зависит от наличия экрана, поэтому
% \begin{equation}
% 	\frac{H_{in}^{(0)}}{H_{in}^{(e)}}=\frac{V_0}{V_e},
% \end{equation}
% где $H_{in}^{(0)}$ -- поле в индикаторном соленоиде без надетого экрана, $H_{in}^{(e)}$ -- поле в индикаторном соленоиде с надетым экраном.

% Очевидно, поле $H_{in}^{(0)}=H_{ext}^{(0)}$. Но поле $H_{in}^{(e)}$ -- это не ослабленное поле $H_{ext}^{(0)}$, потому что при внесении экрана в генераторный соленоид $H_{ext}=H_{ext}^{(e)}=H_{ext}^{(0)}\cdot\frac{L_0}{L_e}$, где $L_0$ -- индуктивность генераторного соленоида без экрана, $L_e$ -- с экраном.

% Значит, если бы при внесенном экране генераторный соленоид создавал поле $H_{ext}^{(0)}$, то внутри экрана бы было поле $H_{in}^{(e)}\cdot\frac{L_e}{L_0}$ (в линейном приближении).

% Отсюда следует, что
% \begin{equation}
% 	|\eta_m|=\frac{H_{ext}^{(0)}}{H_{in}^{(e)}\cdot\frac{L_e}{L_0}}=\frac{V_0}{V_e}\cdot\frac{L_0}{L_e}
% \end{equation}

% Так как мы измеряем напряжение $U$ на резисторе в цепи генераторного соленоида (допустим, что резистор не искажает импеданс: $R\ll \omega L$)

% \begin{equation}
% 	U_e=I_eR, \quad\Rightarrow\quad L_e\sim \frac{u_0}{\omega I_e}=\frac{u_0 R}{\omega U_e}
% \end{equation}
% С другой стороны, 
% \begin{equation}
% 	L_0\sim \frac{u_0}{\omega I_0} = \frac{u_0 R}{\omega U_0}
% \end{equation}
% Тогда окончательно
% \begin{equation}
% 	|\eta_m|=\frac{V_0U_e}{V_eU_0}
% \label{eq:7}
% \end{equation}
% где $V$ и $U$ - соответственно показания вольтметра и осциллографа, индексы $0$ и $e$ относятся соответственно к величинам измеренным без экрана и с экраном.


\newpage
\section{Экспериментальные результаты}
\subsection{Измерение $|\eta_m|$ латунных и стальных экранов}

При использовании каждого экрана производилась подстройка напряжения на генераторном соленоиде, чтобы при отсутствии
экрана значение напряжения на вольтметре было равно $V_0=1000$ мВ для всех экспериментов.

% \begin{table}[h!]
% 	\caption{Измерение экранирования латунными экранами}
% 	\label{tab:6s1}
% 	\vspace{1em}
% 	\centering
% 	\includegraphics[width=\textwidth]{tables/table1}
% \end{table}

% У стальных экранов некоторые измерения не были произведены полностью, ввиду сильного падения $V_e$ и появления шумов,
% искажающих результаты (шум больше точности измерения).

% \begin{table}[h!]
% 	\caption{Измерение экранирования стальными экранами}
% 	\label{tab:6s1}
% 	\vspace{1em}
% 	\centering
% 	\includegraphics[width=\textwidth]{tables/table2}
% \end{table}

По результатам измерений (см. таблицы \ref{tab:lead} и \ref{tab:steel}) для всех частот и экранов рассчитан $|\eta_m|$ и построены графики в логарифмическом
масштабе.

% На рисунке \ref{fig:all} (см. стр. \pageref{fig:all}) приведены шесть графиков для каждого экрана. 

% \begin{figure}[H]
% 	% \vspace{-15pt}
% 	\centering
% 	\includegraphics[scale=1]{imgs/graphs/eta}
% 	\caption{Результаты эксперимента для трех латунных и трех стальных экранов}
% 	\label{fig:all}
% \end{figure}

\subsection{Латунные экраны}

\begin{figure}[H]
	% \vspace{-10pt}
	\centering
	\includegraphics[width=0.95\linewidth]{fig/lat}
	\caption{Экспериментальные и теоретические (пунктир) графики для экранов из латуни. Значения для теоретических
	графиков: $\sigma \simeq 1.5 \cdot 10^{17}c^{-1}, \mu \simeq 1$.}
	\label{fig:figure3}
	% \vspace{-10pt}
\end{figure}

Для построения теоретических графиков, необходимо определить границы применимости формул \eqref{eq:2} и \eqref{eq:4}.
Решая уравнение вида $\delta(f^*) = d$, можно найти такую частоту $f^*$, что при частотах $f>f^*$, можно считать справедливой формулу
\eqref{eq:2}, а при $f<f^*$ - формулу \eqref{eq:4}.

Значения $f^*$ для латуни при разных значениях $d$:
\begin{table}[H]
	\centering
	\begin{tabular}{|c|c|c|c|}
	\hline
	$d,$ мм & 2 & 5 & 10 \\ \hline
	$f^*$, Гц  & 4000  & 628  & 156 \\ \hline
	\end{tabular}
	% \caption{}
	% \label{tab:my-table}
\end{table}





% \begin{figure}[H]
% 	% \vspace{-10pt}
% 	\centering
% 	\includegraphics[scale=1]{imgs/graphs/delta}
% 	\caption{Разграничение применимости формул толщиной скин-слоя $\delta(f)$}
% 	\label{fig:figure3}

% \end{figure}

Принимая в качестве модели цилиндрического экрана сферический слой той же толщины $d$ и с тем же объемом внутренней
полости $V=(4\pi/3)(a-d)^3=\pi R^2h$ (отсюда, ввиду $a\gg d$, имеем $a\cong (3R^2h/4)^{1/3}$), построили для исследуемых
экранов графики теоретической зависимости $|\eta_m(f)|$.

% Хорошее качественное совпадение наблюдается в области частот до 6 кГц.
Как видно из \ref{fig:figure3}, наблюдается достаточно хорошее совпадение теории и эксперимента, однако теоретические
кривые нарастают быстрее экспериментальных с ростом частоты.

% \newpage
\subsection{Оценка $\mu$ для стальных экранов по результатам измерений}

Для стальных экранов почти всюду выполняется $\delta \ll d$, поэтому оценка производится из формулы

\begin{gather}
	\eta_m = \frac16\qty[(1-i)\frac{\mu \delta}{a} + 3 + (1+i)\frac{a}{\mu \delta}]\exp\qty[(1+i)\frac{d}{\delta}]
\end{gather}

Взяв модуль от этого выражения, получим:
\begin{gather}
	|\eta_m| = \frac{\exp\qty[\frac{d}{\delta}]}{6}\sqrt{\qty(\frac{\mu \delta}{a} +3 + \frac{a}{\mu \delta} )^2 + \qty(\frac{a}{\mu \delta} - \frac{\mu \delta}{a})^2}
	\label{eq:8}
\end{gather}

Для определения $\mu$ по известным значениям $|\eta_m|$ использовался графический метод для уравнения \eqref{eq:8}.
Для наглядности, стрроились правая и левая части следующего вида:
\begin{equation}
	\begin{aligned}
		&LHS: y(\mu) = |\eta_m(f)|\exp\qty[-\frac{d}{\delta}]\\ 
		&RHS: x(\mu) = \frac16 \sqrt{\qty(\frac{\mu \delta}{a} +3 + \frac{a}{\mu \delta} )^2 + \qty(\frac{a}{\mu \delta} - \frac{\mu \delta}{a})^2}
	\end{aligned}
		\label{eq:9}
\end{equation}

% \begin{figure}[H]
% 	\centering
% 	\includegraphics[width = 0.95\linewidth]{imgs/graphs/mu.png}
% 	\caption{Графическое решение для значения $\mu$ для стали}
% 	\label{fig:mu}
% \end{figure}

Для 2 мм полученное таким методом значение на частоте 500 Гц дает $\mu=152$.
% На графике (см. рис \ref{fig:eta_wt_steel}, стр.\pageref{fig:eta_wt_steel}) хорошо видно, что действительно это значение дает численное решение этого уравнения, и теоретический график проходит через практическую точку.

Для 5 мм (500 Гц) $\mu=154$, для 10 мм (200 Гц) $\mu=126$.

% Экспериментальные точки подбирались таким образом, чтобы рассчитанная из них $\mu$ давала теоретические графики, наиболее хорошим образом описывающие экспериментальные кривые, хотя бы в диапазоне не очень больших частот.

Расхождение теоретического графика (который уходит в значительно большие по сравнению с практическими $|\eta_m|$)  и
практического, можно объяснить частотным насыщением магнитной проницаемости стали: из-за инертной природы перестроения
доменной структуры, она не успевает изменяться вслед за частотой поля, и $\mu$ начинает падать с ростом частоты.

\begin{figure}[H]
	\vspace{-20pt}
	\centering
	\includegraphics[width=0.95\linewidth]{fig/st}
	\caption{Экспериментальные и теоретические (пунктир) графики для экранов из стали.}
	\label{fig:eta_wt_steel}
\end{figure}
\section{Результаты}
\begin{enumerate}
	\item В работе было исследовано явление экранирования переменного магнитного поля стальными и латунными экранами. 

	\item Произведен расчет и сопоставление экранирующих свойств латунных экранов с экспериментальными с помощью модели сферического слоя. Выявлено хорошее совпадение теории с практикой до $f=6$ кГц. 

	\item Численными методами найдены $\mu$ для стальных экранов, дающие наиболее адекватное соответствие теоретических графиков практическим: $\mu=153,140,130$ для 2,5,10 мм экранов. В этом случае теория дает качественное соответствие вплоть на частотах $f \simeq 1$ кГц.
\end{enumerate}

\subsection{Таблицы измерений}
\begin{table}[H]
	\centering
	\begin{tabular}{|c|c|c|c|c|c|c|c|c|}
	\hline
	\multicolumn{3}{|c|}{Характеристика экрана}& \multicolumn{2}{c|}{Латунь 2 мм} & \multicolumn{2}{c|}{Латунь 5 мм} & \multicolumn{2}{c|}{Латунь 10 мм} \\ \hline
	\multicolumn{1}{|c|}{$f$, Гц} & \multicolumn{1}{c|}{$V_0$, мВ} & \multicolumn{1}{c|}{U, мВ} &	\multicolumn{1}{c|}{$V_e$, мВ}&U, мВ  & $V_e$, мВ &U, мВ&$V_e$, мВ&U, дел\\
	\hline
	20   &1000&2100&920&2680&960&2550&870  &2590\\ \hline
	50   &1000&1410&840&1530&880&1480&700  &1520\\ \hline
	100  &1000&560 &690&820 &600&820 &500  &820\\ \hline
	200  &1000&500 &650&540 &480&540 &350  &600\\ \hline
	500  &1000&340 &500&420 &300&480 &180  &560\\ \hline
	1000 &1000&150 &380&196 &180&208 &100  &280\\ \hline
	2000 &1000&112 &240&152 &80 &168 &33   &224\\ \hline
	5000 &1000&80  &90 &136 &28 &146 &3.6  &220\\ \hline
	10000&800 &54   &54 &48  &7.4&66 &0.22 &142\\ \hline
	\end{tabular}
	\caption{Экспериментальные данные для латунных экранов}
	\label{tab:lead}
\end{table}

\begin{table}[H]
	\centering
	\begin{tabular}{|c|c|c|c|c|c|c|c|c|}
	\hline
	\multicolumn{3}{|c|}{Характеристика экрана}& \multicolumn{2}{c|}{Сталь 2 мм} & \multicolumn{2}{c|}{Сталь 5 мм} & \multicolumn{2}{c|}{Сталь 10 мм} \\ \hline
	\multicolumn{1}{|c|}{$f$, Гц} & \multicolumn{1}{c|}{$V_0$, мВ} & \multicolumn{1}{c|}{U, мВ} &	\multicolumn{1}{c|}{$V_e$, мВ}&U, мВ & $V_e$, мВ &U, мВ&$V_e$, мВ&U, мВ\\
	\hline
	20   &1000&2100&48  &2880&21&2790  &3    &2720\\ \hline
	50   &1000&1410&69  &1150&12&1120  &1.8  &520 \\ \hline
	100  &1000&560 &48  &500 &6.3&420  &0.31 &192 \\ \hline
	200  &1000&500 &43  &300 &2.8&300  &0.04 &152 \\ \hline
	500  &1000&340 &24  &200 &0.28&200 &0.008&88  \\ \hline
	1000 &1000&150 &8   &100 &0.08&88  & --  & -- \\ \hline
	2000 &1000&112 &3.4 &92  & -- & -- & --  & -- \\ \hline
	5000 &1000&80  &0.22&92  & -- & -- & --  & -- \\ \hline
	10000&800 &54  & -- & -- & -- & -- & --  & -- \\ \hline
	\end{tabular}
	\caption{Экспериментальные данные для стальных экранов}
	\label{tab:steel}
\end{table}


\end{document}
